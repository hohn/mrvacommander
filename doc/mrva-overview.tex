\documentclass[11pt]{article}

% Load the geometry package to set margins
\usepackage[margin=1cm]{geometry}

% Load CM Bright for math
\usepackage{amsmath}  % Standard math package
\usepackage{amssymb}  % Additional math symbols
\usepackage{cmbright} % Sans-serif math font that complements Fira Sans

% Font configuration
% \usepackage{bera} 
% or
% Load Fira Sans for text
\usepackage{fontspec}
\setmainfont{Fira Sans}  % System-installed Fira Sans
\renewcommand{\familydefault}{\sfdefault}  % Set sans-serif as default

% pseudo-code with math
\usepackage{listings}
\usepackage{float}
\usepackage{xcolor}
\usepackage{colortbl}
% Set TT font
% \usepackage{inconsolata}
% or
\setmonofont{IBMPlexMono-Light}
% Define custom settings for listings
\lstset{
    language=Python,
    basicstyle=\ttfamily\small,        % Monospaced font
    commentstyle=\itshape\color{gray}, % Italic and gray for comments
    keywordstyle=\color{blue},         % Keywords in blue
    stringstyle=\color{red},           % Strings in red
    mathescape=true,                   % Enable math in comments
    breaklines=true,                   % Break long lines
    numbers=left,                      % Add line numbers
    numberstyle=\tiny\color{gray},     % Style for line numbers
    frame=single,                      % Add a frame around the code
}

\usepackage{newfloat}  % Allows creating custom float types

% Define 'listing' as a floating environment
\DeclareFloatingEnvironment[
    fileext=lol,
    listname=List of Listings,
    name=Listing
]{listing}

% To prevent floats from moving past a section boundary but still allow some floating:
\usepackage{placeins}
% used with \FloatBarrier 

\usepackage[utf8]{inputenc}
\usepackage[T1]{fontenc}
\usepackage{graphicx}
\usepackage{longtable}
\usepackage{wrapfig}
\usepackage{rotating}
\usepackage[normalem]{ulem}
\usepackage{amsmath}
\usepackage{amssymb}
\usepackage{capt-of}
\usepackage{hyperref}
\usepackage{algorithm}
\usepackage{algpseudocode}

% Title, Author, and Date (or Report Number)
\title{MRVA for CodeQL}
\author{Michael Hohn}
\date{Technical Report 20250224}

\hypersetup{
 pdfauthor={Michael Hohn},
 pdftitle={MRVA for CodeQL},
 pdfkeywords={},
 pdfsubject={},
 pdfcreator={Emacs 29.1},
 pdflang={English}}

\begin{document}

\maketitle
\tableofcontents

\section{MRVA System Architecture Summary}

The MRVA system is organized as a collection of services. On the server side, the
system is containerized using Docker and comprises several key components:
\begin{itemize}
    \item {\textbf{Server}}: Acts as the central coordinator.
    \item \textbf{Agents}: One or more agents that execute tasks.
    \item \textbf{RabbitMQ}: Handles messaging between components.
    \item \textbf{MinIO}: Provides storage for both queries and results.
    \item \textbf{HEPC}: An HTTP endpoint that hosts and serves CodeQL databases.
\end{itemize}

On the client side, users can interact with the system in two ways:
\begin{itemize}
    \item {\textbf{VSCode-CodeQL}}: A graphical interface integrated with Visual Studio Code.
    \item \textbf{gh-mrva CLI}: A command-line interface that connects to the server in a similar way.
\end{itemize}

This architecture enables a robust and flexible workflow for code analysis, combining a containerized back-end with both graphical and CLI front-end tools.

The full system details can be seen in the source code.  This document provides an
overview.

\section{Distributed Query Execution in MRVA}

\subsection{Execution Overview}

The \textit{MRVA system} is a distributed platform for executing \textit{CodeQL
queries} across multiple repositories using a set of worker agents. The system is
{containerized} and built around a set of core services:

\begin{itemize}
    \item \textbf{Server}: Coordinates job distribution and result aggregation.
    \item \textbf{Agents}: Execute queries independently and return results.
    \item \textbf{RabbitMQ}: Handles messaging between system components.
    \item \textbf{MinIO}: Stores query inputs and execution results.
    \item \textbf{HEPC}: Serves CodeQL databases over HTTP.
\end{itemize}

Clients interact with MRVA via \texttt{VSCode-CodeQL} (a graphical interface) or
\texttt{gh-mrva CLI} (a command-line tool), both of which submit queries to the
server.

The execution process follows a structured workflow:

\begin{enumerate}
    \item A client submits a set of queries $\mathcal{Q}$ targeting a repository
      set $\mathcal{R}$.
    \item The server enqueues jobs and distributes them to available agents.
    \item Each agent retrieves a job, executes queries against its assigned repository, and accumulates results.
    \item The agent sends results back to the server, which then forwards them to the client.
\end{enumerate}

This full round-trip can be expressed as:

\begin{equation}
    \text{Client} \xrightarrow{\mathcal{Q}} \text{Server}
    \xrightarrow{\text{enqueue}} 
    \text{Queue} \xrightarrow{\text{dispatch}} \text{Agent}
    \xrightarrow{\mathcal{Q}(\mathcal{R}_i)}
    \text{Server} \xrightarrow{\mathcal{Q}(\mathcal{R}_i} \text{Client}
\end{equation}

where the Client submits queries to the Server, which enqueues jobs in the
Queue. Agents execute the queries, returning results $\mathcal{Q}(\mathcal{R}_i)$
to the Server and ultimately back to the Client.

A more rigorous description of this is in section \ref{sec:full-round-trip}.

\subsection{System Structure Overview}

This design allows for scalable and efficient query execution across multiple
repositories, whether on a single machine or a distributed cluster. The key idea
is that both setups follow the same structural approach:

\begin{itemize}
\item \textbf{Single machine setup:}
  \begin{itemize}
  \item Uses \textit{at least 5 Docker containers} to manage different
    components of the system.
  \item The number of \textit{agent containers} (responsible for executing
    queries) is constrained by the available \textit{RAM and CPU cores}.
  \end{itemize}
  
\item \textbf{Cluster setup:}
  \begin{itemize}
  \item Uses \textit{at least 5 virtual machines (VMs) and / or Docker containers}.
  \item The number of \textit{agent VMs} is limited by \textit{network bandwidth
      and available resources} (e.g., distributed storage and inter-node communication
    overhead).
  \end{itemize}
\end{itemize}

Thus:
\begin{itemize}
    \item The {functional architecture is identical} between the single-machine and cluster setups.
    \item The {primary difference} is in \textit{scale}:
    \begin{itemize}
        \item A single machine is limited by \textit{local CPU and RAM}.
        \item A cluster is constrained by \textit{network and inter-node coordination overhead} but allows for higher overall compute capacity.
    \end{itemize}
\end{itemize}


\subsection{Messages and their Types}
\label{sec:msg-types}
The following table enumerates the types (messages) passed from Client to Server.

\begin{longtable}{|p{5cm}|p{5cm}|p{5cm}|}
\hline
\rowcolor{gray!20} \textbf{Type Name} & \textbf{Field} & \textbf{Type} \\
\hline
\endfirsthead

\hline
\rowcolor{gray!20} \textbf{Type Name} & \textbf{Field} & \textbf{Type} \\
\hline
\endhead

\hline
\endfoot

\hline
\endlastfoot

ServerState & NextID & () $\rightarrow$ int \\
& GetResult & JobSpec $\rightarrow$ IO (Either Error AnalyzeResult) \\
& GetJobSpecByRepoId & (int, int) $\rightarrow$ IO (Either Error JobSpec) \\
& SetResult & (JobSpec, AnalyzeResult) $\rightarrow$ IO () \\
& GetJobList & int $\rightarrow$ IO (Either Error \textbf{[AnalyzeJob]}) \\
& GetJobInfo & JobSpec $\rightarrow$ IO (Either Error JobInfo) \\
& SetJobInfo & (JobSpec, JobInfo) $\rightarrow$ IO () \\
& GetStatus & JobSpec $\rightarrow$ IO (Either Error Status) \\
& SetStatus & (JobSpec, Status) $\rightarrow$ IO () \\
& AddJob & AnalyzeJob $\rightarrow$ IO () \\

\hline
JobSpec & sessionID & int \\
& nameWithOwner & string \\

\hline
AnalyzeResult & spec & JobSpec \\
& status & Status \\
& resultCount & int \\
& resultLocation & ArtifactLocation \\
& sourceLocationPrefix & string \\
& databaseSHA & string \\

\hline
ArtifactLocation & Key & string \\
& Bucket & string \\

\hline
AnalyzeJob & Spec & JobSpec \\
& QueryPackLocation & ArtifactLocation \\
& QueryLanguage & QueryLanguage \\

\hline
QueryLanguage &  & string \\

\hline
JobInfo & QueryLanguage & string \\
& CreatedAt & string \\
& UpdatedAt & string \\
& SkippedRepositories & SkippedRepositories \\

\hline
SkippedRepositories & AccessMismatchRepos & AccessMismatchRepos \\
& NotFoundRepos & NotFoundRepos \\
& NoCodeqlDBRepos & NoCodeqlDBRepos \\
& OverLimitRepos & OverLimitRepos \\

\hline
AccessMismatchRepos & RepositoryCount & int \\
& Repositories & \textbf{[Repository]} \\

\hline
NotFoundRepos & RepositoryCount & int \\
& RepositoryFullNames & \textbf{[string]} \\

\hline
Repository & ID & int \\
& Name & string \\
& FullName & string \\
& Private & bool \\
& StargazersCount & int \\
& UpdatedAt & string \\

\end{longtable}


\section{Symbols and Notation}
\label{sec:orgb695d5a}

We define the following symbols for entities in the system:

\begin{center}
    \begin{tabular}{lll}
        Concept                                                                       & Symbol                            & Description                                                         \\[0pt]
        \hline
        \href{vscode://file//Users/hohn/work-gh/mrva/gh-mrva/README.org:39:1}{Client} & \(C\)                             & The source of the query submission                                  \\[0pt]
        Server                                                                        & \(S\)                             & Manages job queue and communicates results back to the client       \\[0pt]
        Job Queue                                                                     & \(Q\)                             & Queue for managing submitted jobs                                   \\[0pt]
        Agent                                                                         & \(\alpha\)                        & Independently polls, executes jobs, and accumulates results         \\[0pt]
        Agent Set                                                                     & \(A\)                             & The set of all available agents                                     \\[0pt]
        Query Suite                                                                   & \(\mathcal{Q}\)                   & Collection of queries submitted by the client                       \\[0pt]
        Repository List                                                               & \(\mathcal{R}\)                   & Collection of repositories                                          \\[0pt]
        \(i\)-th Repository                                                           & \(\mathcal{R}_i\)                 & Specific repository indexed by \(i\)                                \\[0pt]
        \(j\)-th Query                                                                & \(\mathcal{Q}_j\)                 & Specific query from the suite indexed by \(j\)                      \\[0pt]
        Query Result                                                                  & \(r_{i,j,k_{i,j}}\)               & \(k_{i,j}\)-th result from query \(j\) executed on repository \(i\) \\[0pt]
        Query Result Set                                                              & \(\mathcal{R}_i^{\mathcal{Q}_j}\) & Set of all results for query \(j\) on repository \(i\)              \\[0pt]
        Accumulated Results                                                           & \(\mathcal{R}_i^{\mathcal{Q}}\)   & All results from executing all queries on \(\mathcal{R}_i\)         \\[0pt]
    \end{tabular}
\end{center}


\section{Full Round-Trip Representation}
\label{sec:full-round-trip}
The full round-trip execution, from query submission to result delivery, can be summarized as:

\[
    C \xrightarrow{\mathcal{Q}} S \xrightarrow{\text{enqueue}} Q
    \xrightarrow{\text{poll}}
    \alpha \xrightarrow{\mathcal{Q}(\mathcal{R}_i)} S \xrightarrow{\mathcal{R}_i^{\mathcal{Q}}} C
\]

\begin{itemize}
    \item \(C \to S\): Client submits a query suite \(\mathcal{Q}\) to the server.
    \item \(S \to Q\): Server enqueues the query suite \((\mathcal{Q}, \mathcal{R}_i)\) for each repository.
    \item \(Q \to \alpha\): Agent \(\alpha\) polls the queue and retrieves a job.
    \item \(\alpha \to S\): Agent executes the queries and returns the accumulated results \(\mathcal{R}_i^{\mathcal{Q}}\) to the server.
    \item \(S \to C\): Server sends the complete result set \(\mathcal{R}_i^{\mathcal{Q}}\) for each repository back to the client.
\end{itemize}

\section{Result Representation}

For the complete collection of results across all repositories and queries:
\[
  \mathcal{R}^{\mathcal{Q}} = \bigcup_{i=1}^{N} \bigcup_{j=1}^{M}
  \left\{ r_{i,j,1}, r_{i,j,2}, \dots, r_{i,j,k_{i,j}} \right\}
\]

where:
\begin{itemize}
    \item \(N\) is the total number of repositories.
    \item \(M\) is the total number of queries in \(\mathcal{Q}\).
    \item \(k_{i,j}\) is the number of results from executing query
      \(\mathcal{Q}_j\)
      on repository \(\mathcal{R}_i\).
\end{itemize}

An individual result from the \(i\)-th repository, \(j\)-th query, and \(k\)-th result is:
\[
    r_{i,j,k}
\]



\[
    C \xrightarrow{\mathcal{Q}} S \xrightarrow{\text{enqueue}} Q \xrightarrow{\text{dispatch}} \alpha \xrightarrow{\mathcal{Q}(\mathcal{R}_i)} S \xrightarrow{r_{i,j}} C
\]

Each result can be further indexed to track multiple repositories and result sets.

\section{Execution Loop in Pseudo-Code}
\begin{listing}[h] % h = here, t = top, b = bottom, p = page of floats
    \caption{Distributed Query Execution Algorithm}

    \begin{lstlisting}[language=Python]
# Distributed Query Execution with Agent Polling and Accumulated Results

# Initialization
$\mathcal{R}$ = set()  # Repository list
$Q$ = []  # Job queue
$A$ = set()  # Set of agents
$\mathcal{R}_i^{\mathcal{Q}}$ = {}  # Result storage for each repository

# Initialize result sets for each repository
for $R_i$ in $\mathcal{R}$:
    $\mathcal{R}_i^{\mathcal{Q}} = \{\}$  # Initialize empty result set

# Enqueue the entire query suite for all repositories
for $R_i$ in $\mathcal{R}$:
    $Q$.append(($\mathcal{Q}$, $R_i$))  # Enqueue $(\mathcal{Q}, \mathcal{R}_i)$ pair

# Processing loop while there are jobs in the queue
while $Q \neq \emptyset$:
    # Agents autonomously poll the queue
    for $\alpha$ in $A$:
        if $\alpha$.is_available():
            $(\mathcal{Q}, \mathcal{R}_i)$ = $Q$.pop(0)  # Agent polls a job

            # Agent execution begins
            $\mathcal{R}_i^{\mathcal{Q}} = \{\}$  # Initialize results for repository $R_i$

            for $\mathcal{Q}_j$ in $\mathcal{Q}$:
                # Execute query $\mathcal{Q}_j$ on repository $\mathcal{R}_i$
                $r_{i,j,1}, \dots, r_{i,j,k_{i,j}}$ = $\alpha$.execute($\mathcal{Q}_j$, $R_i$)

                # Store results for query $j$
                $\mathcal{R}_i^{\mathcal{Q}_j} = \{r_{i,j,1}, \dots, r_{i,j,k_{i,j}}\}$

                # Accumulate results
                $\mathcal{R}_i^{\mathcal{Q}} = \mathcal{R}_i^{\mathcal{Q}} \cup \mathcal{R}_i^{\mathcal{Q}_j}$

            # Send all accumulated results back to the server
            $\alpha$.send_results($S$, ($\mathcal{Q}$, $R_i$, $\mathcal{R}_i^{\mathcal{Q}}$))

            # Server sends results for $(\mathcal{Q}, \mathcal{R}_i)$ back to the client
            $S$.send_results_to_client($C$, ($\mathcal{Q}$, $R_i$, $\mathcal{R}_i^{\mathcal{Q}}$))
\end{lstlisting}
\end{listing}
\FloatBarrier

\section{Execution Loop in Pseudo-Code, declarative}
\begin{listing}[h] % h = here, t = top, b = bottom, p = page of floats
  \caption{Distributed Query Execution Algorithm}

\begin{lstlisting}[language=Python]
# Distributed Query Execution with Agent Polling and Accumulated Results

# Define initial state
$\mathcal{R}$: set          # Set of repositories
$\mathcal{Q}$: set          # Set of queries
A: set          # Set of agents
Q: list         # Queue of $(\mathcal{Q}, \mathcal{R}_i)$ pairs
$\mathcal{R}_{\text{results}}$: dict = {}  # Mapping of repositories to their accumulated query results

# Initialize result sets for each repository
$\mathcal{R}_{\text{results}}$ = {$\mathcal{R}_i$: set() for $\mathcal{R}_i$ in $\mathcal{R}$}

# Define job queue as an immutable mapping
Q = [($\mathcal{Q}$, $\mathcal{R}_i$) for $\mathcal{R}_i$ in $\mathcal{R}$]

# Processing as a declarative iteration over the job queue
def execute_queries(agents, job_queue, repository_results):
    def available_agents():
        return {$\alpha$ for $\alpha$ in agents if $\alpha$.is_available()}

    def process_job($\mathcal{Q}$, $\mathcal{R}_i$, $\alpha$):
        results = {$\mathcal{Q}_j$: $\alpha$.execute($\mathcal{Q}_j$, $\mathcal{R}_i$) for $\mathcal{Q}_j$ in $\mathcal{Q}$}
        return $\mathcal{R}_i$, results

    def accumulate_results($\mathcal{R}_{\text{results}}$, $\mathcal{R}_i$, query_results):
        return {**$\mathcal{R}_{\text{results}}$, $\mathcal{R}_i$: $\mathcal{R}_{\text{results}}$[$\mathcal{R}_i$] | set().union(*query_results.values())}

    while job_queue:
        active_agents = available_agents()
        for $\alpha$ in active_agents:
            $\mathcal{Q}$, $\mathcal{R}_i$ = job_queue[0]  # Peek at the first job
            _, query_results = process_job($\mathcal{Q}$, $\mathcal{R}_i$, $\alpha$)
            repository_results = accumulate_results(repository_results, $\mathcal{R}_i$, query_results)

            $\alpha$.send_results(S, ($\mathcal{Q}$, $\mathcal{R}_i$, repository_results[$\mathcal{R}_i$]))
            S.send_results_to_client(C, ($\mathcal{Q}$, $\mathcal{R}_i$, repository_results[$\mathcal{R}_i$]))

        job_queue = job_queue[1:]  # Move to the next job

    return repository_results

# Execute the distributed query process
$\mathcal{R}_{\text{results}}$ = execute_queries(A, Q, $\mathcal{R}_{\text{results}}$)
\end{lstlisting}



\end{listing}
\FloatBarrier

\section{Execution Loop in Pseudo-Code, algorithmic}
\begin{algorithm}
    \caption{Distribute a set of queries $\mathcal{Q}$ across repositories
      $\mathcal{R}$ using agents $A$}
    \begin{algorithmic}[1] % Line numbering enabled
        \Procedure{DistributedQueryExecution}{$\mathcal{Q}, \mathcal{R}, A$}

        \ForAll{$\mathcal{R}_i \in \mathcal{R}$}
        \Comment{Initialize result sets for each repository and query}
        \State $\mathcal{R}_i^{\mathcal{Q}} \gets \left\{ \, \right\}$
        \EndFor

        \State $Q \gets \left\{ \, \right\}$ \Comment{Initialize empty job queue}

        \ForAll{$\mathcal{R}_i \in \mathcal{R}$}
        \Comment{Enqueue the entire query suite across all repositories}
        \State $S \xrightarrow{\text{enqueue}(\mathcal{Q}, \mathcal{R}_i)} Q$
        \EndFor

        \While{$Q \neq \emptyset$}
        \Comment{Agents poll the queue for available jobs}

        \ForAll{$\alpha \in A$ \textbf{where} $\alpha$ \text{is available}}
        \State $\alpha \xleftarrow{\text{poll}(Q)}$ \Comment{Agent autonomously retrieves a job}

        % --- Begin Agent Execution Block ---
        \State \textbf{(Agent Execution Begins)}

        \State $\mathcal{R}_i^{\mathcal{Q}} \gets \left\{ \, \right\}$ \Comment{Initialize result set for this repository}

        \ForAll{$\mathcal{Q}_j \in \mathcal{Q}$}
        \State $\mathcal{R}_i^{\mathcal{Q}_j} \gets \left\{ r_{i,j,1}, r_{i,j,2}, \dots, r_{i,j,k_{i,j}} \right\}$
        \Comment{Collect results for query $j$ on repository $i$}

        \State $\mathcal{R}_i^{\mathcal{Q}} \gets \mathcal{R}_i^{\mathcal{Q}}
            \cup \mathcal{R}_i^{\mathcal{Q}_j}$
        \Comment{Accumulate results}
        \EndFor

        \State $\alpha \xrightarrow{(\mathcal{Q}, \mathcal{R}_i, \mathcal{R}_i^{\mathcal{Q}})} S$
        \Comment{Agent sends all accumulated results back to server}

        \State \textbf{(Agent Execution Ends)}
        % --- End Agent Execution Block ---

        \State $S \xrightarrow{(\mathcal{Q}, \mathcal{R}_i, \mathcal{R}_i^{\mathcal{Q}})} C$
        \Comment{Server sends results for repository $i$ back to the client}

        \EndFor

        \EndWhile

        \EndProcedure
    \end{algorithmic}
\end{algorithm}

\FloatBarrier


\end{document}

%%% Local Variables:
%%% mode: LaTeX
%%% TeX-master: t
%%% TeX-engine: luatex
%%% TeX-command-extra-options: "-synctex=1 -shell-escape -interaction=nonstopmode"
%%% End:
